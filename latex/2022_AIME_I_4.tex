\documentclass{article}
\usepackage{geometry}
\usepackage{amsmath}
\usepackage{parskip}
\usepackage{hyperref}

\title{2022 AIME I Problem 4}
\author{bkf2020}
\date{Updated on \today}

\hypersetup{
    urlcolor=blue,
}

\begin{document}
\maketitle

\section*{LICENSE}
Under \href{https://creativecommons.org/publicdomain/zero/1.0/}{CC0}

\section*{Problem Link}
\href{https://artofproblemsolving.com/wiki/index.php/2022\_AIME\_I\_Problems#Problem\_4}{2022 AIME I Problem 4}

\section*{Commentary}
In this problem, $w = e^{\pi i/6}$ and $z = e^{5\pi i/3}$. Thus, we have:
$$i \cdot w^r = z^s \Longrightarrow e^{\pi i/2} \cdot e^{\pi ir/6} = e^{5\pi is/3}$$ from the condition given
in the problem.

I messed up on the test by assuming this meant

$$\frac{\pi i}{2} + \frac{\pi ir}{6} = \frac{5\pi is}{3}$$ instead of 
$$\frac{\pi i}{2} + \frac{\pi ir}{6} = \frac{5\pi is}{3} + 2\pi i k.$$

Now, when I solved the second equation, I simplified it:
\begin{align*}
	\frac{\pi i}{2} + \frac{\pi ir}{6} &= \frac{5\pi is}{3} + 2\pi i k \\
	3\pi i + \pi ir &= 10\pi is + 12\pi ik \\
	3 + r &= 10s + 12 k \\
	3 + r - 10s &= 12k \\
	r + 2s &\equiv 9 \pmod{12}
\end{align*}
I then noticed that $r$ must be $1, 3, 5, 7, 9$ or $11$ mod 12, and I found the conditions on the values of $s$ for each case.

When $r = 9,$ I got that $s \equiv 0, 6\pmod{12}$ and when $s \equiv 0 \pmod{12}$ I counted 9 possible ways, when there were
only 8 possible ways.

In the end, after fixing this, I got the correct answer.

This casework is kind of long, so I may try to find another way.

\end{document}
